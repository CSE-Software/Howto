\documentclass[]{article}
% Added package start here
\usepackage{array}
\usepackage{geometry} % for margin of paper
\geometry{margin=1.2in, bottom=1.5in}
\usepackage{fancyhdr} % for headers and footers
\usepackage{graphicx} % for image
\graphicspath{ {media/} }
\pagenumbering{gobble} % Delete page number
% End here
\usepackage{lmodern}
\usepackage{amssymb,amsmath}
\usepackage{ifxetex,ifluatex}
\usepackage{fixltx2e} % provides \textsubscript
\ifnum 0\ifxetex 1\fi\ifluatex 1\fi=0 % if pdftex
  \usepackage[T1]{fontenc}
  \usepackage[utf8]{inputenc}
\else % if luatex or xelatex
  \ifxetex
    \usepackage{mathspec}
  \else
    \usepackage{fontspec}
  \fi
  \defaultfontfeatures{Ligatures=TeX,Scale=MatchLowercase}
\fi
% use upquote if available, for straight quotes in verbatim environments
\IfFileExists{upquote.sty}{\usepackage{upquote}}{}
% use microtype if available
\IfFileExists{microtype.sty}{%
\usepackage{microtype}
\UseMicrotypeSet[protrusion]{basicmath} % disable protrusion for tt fonts
}{}
\usepackage{hyperref}
\hypersetup{unicode=true,
            pdfborder={0 0 0},
            breaklinks=true}
\urlstyle{same}  % don't use monospace font for urls
\IfFileExists{parskip.sty}{%
\usepackage{parskip}
}{% else
\setlength{\parindent}{0pt}
\setlength{\parskip}{6pt plus 2pt minus 1pt}
}
\setlength{\emergencystretch}{3em}  % prevent overfull lines
\providecommand{\tightlist}{%
  \setlength{\itemsep}{0pt}\setlength{\parskip}{0pt}}
\setcounter{secnumdepth}{0}
% Redefines (sub)paragraphs to behave more like sections
\ifx\paragraph\undefined\else
\let\oldparagraph\paragraph
\renewcommand{\paragraph}[1]{\oldparagraph{#1}\mbox{}}
\fi
\ifx\subparagraph\undefined\else
\let\oldsubparagraph\subparagraph
\renewcommand{\subparagraph}[1]{\oldsubparagraph{#1}\mbox{}}
\fi

\date{}

\begin{document}
% Added information start here
\pagestyle{fancy}
\renewcommand{\headrulewidth}{0pt}
% footer
  \fancyfoot[L]{\footnotesize This material is based upon work supported by the U.S. Department of Energy Office of Science, Advanced Scientific Computing Research and Biological and Environmental Research programs. \begin{flushright} \footnotesize Version 0.2, April 25, 2016 \end{flushright}}
  
\fancypagestyle{empty}{
% header
\fancyhead[C]{\Large {\textbf{How to Add and Improve Testing}}\\ \Large {\textbf{in your CSE Software Project}}\\ \normalsize {The IDEAS Scientific Software Productivity Project} \\ \small {\href{https://ideas-productivity.org/resources/howtos/}{\emph{ideas-productivity.org/resources/howtos/}}}} 
  \fancyhead[R]{\includegraphics[width = 3 cm, height = 1.5 cm]{ideas_Howto}}	
}
\thispagestyle{empty}
\textbf{\newline}
\textbf{\newline}

% Added information ends here.

\textbf{Overview:} Adding tests of sufficient coverage and quality
improves confidence in software and makes it easier to change and
extend. Tests should be added to existing code before the code is
changed. Tests should be added to new code before (or while) it is being
written. These tests then become the foundation of a regression test
suite that helps effectively drive future development and improves
long-term sustainability.

\textbf{Target Audience:} CSE software project leaders and developers
who are facing significant refactoring efforts because of hardware
architecture changes or increased demands for multiphysics and
multiscale coupling, and who want to increase the quality and speed of
development and reduce development and maintenance costs.

\textbf{Purpose:} Show how to add quality testing to a project in order
to support efficient modification of existing code or addition of new
code. Show how to add tests to support (1) \textbf{adding a new
feature}, (2) \textbf{fixing a bug}, (3) \textbf{improving the design}
\textbf{and implementation}, or (4) \textbf{optimizing resource usage}.

\textbf{Prerequisites:} First read the document
\href{http://ideas-productivity.org/wordpress/wp-content/uploads/2016/04/IDEAS-TestingWhatAreSoftwareTestingPractices-V0.2.pdf}{\emph{What
Are Software Testing Practices?}} and browse through
\href{http://ideas-productivity.org/wordpress/wp-content/uploads/2016/04/IDEAS-TestingWhatIsDefinitionandCategorizationofTestsforCSESoftware-V0.2.pdf}{\emph{Definition
and Categorization of Tests for CSE Software}}.

\textbf{Steps:}

\begin{enumerate}
\def\labelenumi{\arabic{enumi}.}
\item
  \begin{quote}
  Set up \textbf{automated builds of the code} with high warning levels
  and eliminate all warnings.
  \end{quote}
\item
  \begin{quote}
  \textbf{Select test harness frameworks}
  \end{quote}

  \begin{enumerate}
  \def\labelenumii{\alph{enumii}.}
  \item
    \begin{quote}
    \textbf{Select a system-level test harness} for system-executable
    tests that report results appropriately (e.g., CTest/CDash,
    Jenkins).
    \end{quote}
  \item
    \begin{quote}
    \textbf{Select a unit test harness} to effectively define and run
    finer-grained integration and unit tests (e.g., Google Test,
    pFUnit).
    \end{quote}
  \item
    \begin{quote}
    \textbf{Customize or streamline} system-level and/or unit test
    frameworks for use in your particular project.
    \end{quote}
  \end{enumerate}
\item
  \begin{quote}
  \textbf{Add system-level tests} to protect major user functionality.
  \end{quote}

  \begin{enumerate}
  \def\labelenumii{\alph{enumii}.}
  \item
    \begin{quote}
    Select inputs for several important problem classes and run code to
    produce outputs.
    \end{quote}
  \item
    \begin{quote}
    Set up no-change or verification tests with a system-level test
    harness in order to pin down important behavior.
    \end{quote}
  \end{enumerate}
\item
  \begin{quote}
  \textbf{Add integration and unit tests} (as needed for adding/changing
  code)
  \end{quote}

  \begin{enumerate}
  \def\labelenumii{\alph{enumii}.}
  \item
    \begin{quote}
    \textbf{Incorporate tests} {[}1, 2{]} \textbf{for code to be
    changed}
    \end{quote}

    \begin{itemize}
    \item
      \begin{quote}
      \textbf{Identify change points} for target change or new code.
      \end{quote}
    \item
      \begin{quote}
      \textbf{Find test points} where code behavior can be sensed.
      \end{quote}
    \item
      \begin{quote}
      \textbf{Break dependencies} in order to get the targeted code into
      the unit test harness.
      \end{quote}
    \item
      \begin{quote}
      \textbf{Cover targeted code} to be changed with sufficient
      (characterization) tests.
      \end{quote}
    \end{itemize}
  \item
    \begin{quote}
    \textbf{Add new features or fix bugs with tests} {[}1, 2, 3, 4{]}
    \end{quote}

    \begin{itemize}
    \item
      \begin{quote}
      \textbf{Add new tests} that define desired behavior (feature or
      bug).
      \end{quote}
    \item
      \begin{quote}
      Run new tests and \textbf{verify they fail.}
      \end{quote}
    \item
      \begin{quote}
      Add the minimal code to \textbf{get new tests to pass.}
      \end{quote}
    \item
      \begin{quote}
      \textbf{Refactor} the covered code to clean up and remove
      duplication.
      \end{quote}
    \item
      \begin{quote}
      \textbf{Review} all changes to existing code, new code and new
      tests.
      \end{quote}
    \end{itemize}
  \end{enumerate}
\item
  \begin{quote}
  Select \textbf{code coverage} (e.g., gcov/lcov) and \textbf{memory
  usage error detection} (e.g., valgrind) analysis tools.
  \end{quote}
\item
  \begin{quote}
  Define a set of \textbf{regression test suites}
  \end{quote}

  \begin{enumerate}
  \def\labelenumii{\alph{enumii}.}
  \item
    \begin{quote}
    Define a faster-running \textbf{pre-push regression test suite}
    (e.g., single build with faster running tests) and \textbf{run it
    before every push}.
    \end{quote}
  \item
    \begin{quote}
    Define a more comprehensive \textbf{nightly regression test suite}
    (e.g., builds and all tests on several platforms and compilers, code
    coverage, and memory usage error detection) and \textbf{run every
    night}.
    \end{quote}
  \end{enumerate}
\item
  \begin{quote}
  Have a policy of \textbf{100\% passing pre-push regression tests} and
  work hard to maintain that.
  \end{quote}
\item
  \begin{quote}
  Work to \textbf{fix all failing nighty regression tests} on a
  reasonable schedule.
  \end{quote}
\end{enumerate}

\textbf{FAQs:}

\textbf{Q:} \emph{Why do you need both a system-level and a unit test
harness?}

\textbf{A:} A unit test harness aggregates hundreds of unit and
integration tests into single executables. A system-level test harness
runs these aggregate integration and unit test executables along with
the other system-level acceptance and verification tests and alerts
developers of any failures.

\textbf{Q:} \emph{Why not just add all of the tests for an existing code
and get it over with?}

\textbf{A:} Taking weeks or months (or years) to add sufficient tests
for an entire existing code (that lacks sufficient testing) is not
usually economical or necessary. Tests need to be added to code only
when it is changed (or when adding new code). In that way tests can be
added while regular development work is being done.

\textbf{Q:} \emph{Why demand 100\% passing pre-push regression tests?}

\textbf{A:} This avoids expensive debugging and other investigations
needed to determine whether your changes are breaking failing tests or
not (hard). If all tests pass, then your changes could be breaking them
(easy).

\protect\hypertarget{h.9qg5mj337055}{}{}\textbf{References:}

\protect\hypertarget{h.f53m6pxt4rgc}{}{}{[}1{]} Feathers, Michael.
\emph{Working Effectively with Legacy Code}. Prentice Hall, 2005

\protect\hypertarget{h.f34rnf2pj2lo}{}{}{[}2{]} \emph{Legacy Software
Change Algorithm}:
\href{http://trilinos.org/trac/trilinos/wiki/TribitsLegacySoftwareChangeAlgorithm}{\emph{http://trilinos.org/trac/trilinos/wiki/TribitsLegacySoftwareChangeAlgorithm}}

\protect\hypertarget{h.y70wa5kslzka}{}{}{[}3{]} Beck, Kent. \emph{Test
Driven Development}. Addison Wesley, 2003

\protect\hypertarget{h.yohf9hj1puj4}{}{}{[}4{]} McConnell, Steve. Code
Complete (Second Edition). Microsoft Press, 2004

\protect\hypertarget{h.hhclote3u0dy}{}{}

\protect\hypertarget{h.ksyrwlxj2kmg}{}{}This document was prepared by
Ulrike Yang, Roscoe A. Bartlett, Glenn Hammond, Xiaoye Li, Barry Smith,
and James M. Willenbring.

\end{document}
