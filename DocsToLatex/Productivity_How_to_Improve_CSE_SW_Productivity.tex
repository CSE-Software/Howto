\documentclass[]{article}
% Added package start here
\usepackage{array}
\usepackage{geometry} % for margin of paper
\geometry{margin=1.2in}
\usepackage{fancyhdr} % for headers and footers
\usepackage{graphicx} % for image
\graphicspath{ {media/} }
\pagenumbering{gobble} % Delete page number
% End here
\usepackage{lmodern}
\usepackage{amssymb,amsmath}
\usepackage{ifxetex,ifluatex}
\usepackage{fixltx2e} % provides \textsubscript
\ifnum 0\ifxetex 1\fi\ifluatex 1\fi=0 % if pdftex
  \usepackage[T1]{fontenc}
  \usepackage[utf8]{inputenc}
\else % if luatex or xelatex
  \ifxetex
    \usepackage{mathspec}
  \else
    \usepackage{fontspec}
  \fi
  \defaultfontfeatures{Ligatures=TeX,Scale=MatchLowercase}
\fi
% use upquote if available, for straight quotes in verbatim environments
\IfFileExists{upquote.sty}{\usepackage{upquote}}{}
% use microtype if available
\IfFileExists{microtype.sty}{%
\usepackage{microtype}
\UseMicrotypeSet[protrusion]{basicmath} % disable protrusion for tt fonts
}{}
\usepackage{hyperref}
\hypersetup{unicode=true,
            pdfborder={0 0 0},
            breaklinks=true}
\urlstyle{same}  % don't use monospace font for urls
\usepackage{graphicx,grffile}
\makeatletter
\def\maxwidth{\ifdim\Gin@nat@width>\linewidth\linewidth\else\Gin@nat@width\fi}
\def\maxheight{\ifdim\Gin@nat@height>\textheight\textheight\else\Gin@nat@height\fi}
\makeatother
% Scale images if necessary, so that they will not overflow the page
% margins by default, and it is still possible to overwrite the defaults
% using explicit options in \includegraphics[width, height, ...]{}
\setkeys{Gin}{width=\maxwidth,height=\maxheight,keepaspectratio}
\IfFileExists{parskip.sty}{%
\usepackage{parskip}
}{% else
\setlength{\parindent}{0pt}
\setlength{\parskip}{6pt plus 2pt minus 1pt}
}
\setlength{\emergencystretch}{3em}  % prevent overfull lines
\providecommand{\tightlist}{%
  \setlength{\itemsep}{0pt}\setlength{\parskip}{0pt}}
\setcounter{secnumdepth}{0}
% Redefines (sub)paragraphs to behave more like sections
\ifx\paragraph\undefined\else
\let\oldparagraph\paragraph
\renewcommand{\paragraph}[1]{\oldparagraph{#1}\mbox{}}
\fi
\ifx\subparagraph\undefined\else
\let\oldsubparagraph\subparagraph
\renewcommand{\subparagraph}[1]{\oldsubparagraph{#1}\mbox{}}
\fi

\date{}

\begin{document}
% Added information start here
\pagestyle{fancy}
\renewcommand{\headrulewidth}{0pt}
% footer
  \fancyfoot[L]{\footnotesize This material is based upon work supported by the U.S. Department of Energy Office of Science, Advanced Scientific Computing Research and Biological and Environmental Research programs. \begin{flushright} \footnotesize Version 0.2, April 25, 2016 \end{flushright}}
  
\fancypagestyle{empty}{
% header
\fancyhead[C]{\LARGE {Improving Productivity} \\ \LARGE  {in your CSE SoftwareProject} \\ \normalsize {The IDEAS Scientific Software Productivity Project}} 
  \fancyhead[R]{\includegraphics[width = 1.8 cm, height = 1.2 cm]{ideas}}	
}
\thispagestyle{empty}
\textbf{\newline}
\textbf{\newline}
\textbf{\newline}
% Added information ends here.

%\section{Improving Productivity in your CSE Software
%Project}\label{improving-productivity-in-your-cse-software-project}

%An IDEAS Project Howto

\textbf{Overview:} TBD

\textbf{Target Audience:} CSE SW Project leaders and developers who are
facing significant refactoring efforts due to HW architecture changes,
increased demands for multiphysics and multiscale coupling, and who want
to increase the quality and speed of development and reduces development
and maintenance costs.

\textbf{Purpose:} Show how focusing on improved productivity can lead to
more effective project decisions, giving a team better faster and
cheaper.

\textbf{Prerequisites:} This document is for people who need and want to
do a better job of producing, supporting and maintaining CSE software
products.

\begin{itemize}
\item
  \begin{quote}
  Willingness to invest time and resources into efforts that are
  important but not urgent. Focusing on productivity means investing in
  changes that will have long-term payoff.
  \end{quote}
\item
  \begin{quote}
  Resources (staff, funds) to dedicate to productivity improvement.
  Separate this effort from the mainstream daily activities.
  \end{quote}
\end{itemize}

\textbf{Steps:}

\begin{enumerate}
\def\labelenumi{\arabic{enumi}.}
\item
  \begin{quote}
  Develop a working definition of productivity for your team: Discuss
  and define productivity through discussions with your team. The term
  ``productivity'' is often fuzzy when first presented. Through
  education and discussion come up with a good \emph{working}
  definition. If you cannot define productivity in a meaningful way for
  your team, you will not be able to improve it.
  \end{quote}
\item
  \begin{quote}
  Identify productivity goals for your team:
  \end{quote}
\item
  \begin{quote}
  Define questions that clarify what must be done to reach your goals:
  \end{quote}
\item
  \begin{quote}
  Define metrics that track progress in answering your questions.
  \end{quote}
\item
  \begin{quote}
  Record baseline values for your metrics.
  \end{quote}
\item
  \begin{quote}
  Develop strategies and plans to reach goals by answering questions and
  tracking progress via your metrics.
  \end{quote}
\item
  \begin{quote}
  Execute plan.
  \end{quote}
\item
  \begin{quote}
  Record progress compared to baseline.
  \end{quote}
\end{enumerate}

\textbf{FAQ:} TBD

\protect\hypertarget{h.gjdgxs}{}{}\textbf{References:} TBD

\end{document}
