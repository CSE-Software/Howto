\documentclass[]{article}
% Added package start here
\usepackage{array}
\usepackage{geometry} % for margin of paper
\geometry{margin=1.2in}
\usepackage{fancyhdr} % for headers and footers
\usepackage{graphicx} % for image
\graphicspath{ {media/} }
\pagenumbering{gobble} % Delete page number
% End here
\usepackage{lmodern}
\usepackage{amssymb,amsmath}
\usepackage{ifxetex,ifluatex}
\usepackage{fixltx2e} % provides \textsubscript
\ifnum 0\ifxetex 1\fi\ifluatex 1\fi=0 % if pdftex
  \usepackage[T1]{fontenc}
  \usepackage[utf8]{inputenc}
\else % if luatex or xelatex
  \ifxetex
    \usepackage{mathspec}
  \else
    \usepackage{fontspec}
  \fi
  \defaultfontfeatures{Ligatures=TeX,Scale=MatchLowercase}
\fi
% use upquote if available, for straight quotes in verbatim environments
\IfFileExists{upquote.sty}{\usepackage{upquote}}{}
% use microtype if available
\IfFileExists{microtype.sty}{%
\usepackage{microtype}
\UseMicrotypeSet[protrusion]{basicmath} % disable protrusion for tt fonts
}{}
\usepackage{hyperref}
\hypersetup{unicode=true,
            pdfborder={0 0 0},
            breaklinks=true}
\urlstyle{same}  % don't use monospace font for urls
\IfFileExists{parskip.sty}{%
\usepackage{parskip}
}{% else
\setlength{\parindent}{0pt}
\setlength{\parskip}{6pt plus 2pt minus 1pt}
}
\setlength{\emergencystretch}{3em}  % prevent overfull lines
\providecommand{\tightlist}{%
  \setlength{\itemsep}{0pt}\setlength{\parskip}{0pt}}
\setcounter{secnumdepth}{0}
% Redefines (sub)paragraphs to behave more like sections
\ifx\paragraph\undefined\else
\let\oldparagraph\paragraph
\renewcommand{\paragraph}[1]{\oldparagraph{#1}\mbox{}}
\fi
\ifx\subparagraph\undefined\else
\let\oldsubparagraph\subparagraph
\renewcommand{\subparagraph}[1]{\oldsubparagraph{#1}\mbox{}}
\fi

\date{}

\begin{document}
% Added information start here
\pagestyle{fancy}
\renewcommand{\headrulewidth}{0pt}
% footer
  \fancyfoot[L]{\footnotesize This material is based upon work supported by the U.S. Department of Energy Office of Science, Advanced Scientific Computing Research and Biological and Environmental Research programs. \begin{flushright} \footnotesize Version 0.2, April 25, 2016 \end{flushright}}
  
\fancypagestyle{empty}{
% header
\fancyhead[C]{\LARGE {What is  Good Documentation}\\ \normalsize {The IDEAS Scientific Software Productivity Project} \\ \small {\href{https://ideas-productivity.org/resources/howtos/}{\emph{ideas-productivity.org/resources/howtos/}}}} 
  \fancyhead[R]{\includegraphics[width = 1.8 cm, height = 1.2 cm]{ideas}}	
}
\thispagestyle{empty}
\textbf{\newline}
\textbf{\newline}
\textbf{\newline}
% Added information ends here.

\subsubsection{\texorpdfstring{\textbf{Motivation}: Software
documentation is essential for three
reasons:}{Motivation: Software documentation is essential for three reasons:}}\label{motivation-software-documentation-is-essential-for-three-reasons}

\begin{enumerate}
\def\labelenumi{(\arabic{enumi})}
\item
  \begin{quote}
  To identify the purpose and role of the software and its requirements
  \end{quote}
\item
  \begin{quote}
  To clarify what each component does, what is needed in order to
  maintain it, and how (or whether) it can be reused elsewhere
  \end{quote}
\item
  \begin{quote}
  To provide user support and thus minimize unnecessary handholding of
  users
  \end{quote}
\end{enumerate}

Scientific software has an additional reason: to ensure that the
software is used within its region of validity so that the possibility
of producing spurious scientific results is minimized.

\textbf{Categories:} In the wider
software world a good place to start for a general description of
documentation practices is {[}1{]}. For scientific software many
categories for documentation are listed
below.{Categories: In the wider software world a good place to start for a general description of documentation practices is {[}1{]}. For scientific software many categories for documentation are listed below.}

\begin{itemize}
\item
  \begin{quote}
  \textbf{Requirements:} The objectives of the software in terms of
  target applications and expectations from it in terms of features such
  as extensibility and composability.
  \end{quote}
\item
  \begin{quote}
  \textbf{Models and Algorithms:} Mathematical equations being solved
  and the numerical algorithms being used to solve them. These should
  include the range of validity for the models and algorithms.
  \end{quote}
\item
  \begin{quote}
  \textbf{Design Documents:} Design choices in terms of software
  architecture, data structures, parallelization techniques, and other
  similar relevant details.
  \end{quote}
\item
  \begin{quote}
  \textbf{Users Guide:} Specification of how to use the software, how to
  provide input, how to configure a specific problem to solve, and, if
  applicable, how to customize the software.
  \end{quote}
\item
  \begin{quote}
  \textbf{Reference Manual:} List of the interfaces and/or routines and
  explanation of their functionality. This can be automatically
  generated if the information is embedded in the code.
  \end{quote}
\item
  \begin{quote}
  \textbf{Embedded Documentation:} In nonmathematical software one may
  be able to do away with this kind of documentation and instead write
  code that is self-explanatory. However, a mathematical model or even
  numerical algorithm does not have a straight map to implementation.
  Inline documentation is the way to provide this map and is therefore
  indispensable in scientific software.
  \end{quote}
\item
  \begin{quote}
  \textbf{Readme Files:} These descriptive files are typically included
  with the source code of the software. They are often used to describe
  the section of the code (usually within the directory) where they
  reside.
  \end{quote}
\item
  \begin{quote}
  \textbf{Developers Guide:} If the software is extensible, a developers
  guide is helpful for users to extend the software for their own
  purposes.
  \end{quote}
\item
  \begin{quote}
  \textbf{Installation Guide:} If the software has any component or
  dependencies that need installation prior to use, they should be
  described explicitly in some document. It can be a simple text file or
  a more complete guide, depending on the complexity of installation.
  \end{quote}
\item
  \begin{quote}
  \textbf{Process Documentation:} For any nontrivial software effort
  with even a few developers, documenting the software process is
  useful. In large efforts with changing and/or transient developers,
  such documentation is necessary. This kind of documentation may
  include definitions and practices such as licensing, code
  contribution, verification, coding standards, and code reviews.
  \end{quote}
\item
  \begin{quote}
  \textbf{Tutorials:} These consist of the entire necessary input to set
  up, run, and (as appropriate) visualize the output of software. They
  may include source code (for a library) and/or input files (for an
  application code). Often tutorials are the most useful documentation
  for users and can double as regression tests.
  \end{quote}
\end{itemize}

\textbf{Trade-offs:} Documentation is the
least immediately productive aspect of code development. In theory a
code can be developed and perform perfectly without any documentation.
In reality, for any nontrivial code, remembering all the details is
impossible. Lack of documentation will eventually lead to the code's not
meeting its design and performance goals. Arguably, however, generating
documentation involves an overhead. Therefore, each team should decide
how much documentation is optimal for them. For any scientific software,
embedded documentation (comments inlined in the code) is necessary:
developers can, over time, forget why they wrote what they did. But for
all others the teams should weigh their options. For example, a small
team with only internal users may opt to spend all their documentation
efforts in specifications of models and algorithms, whereas the
developers of a community code may want to spend greater effort in
producing a users guide. Many teams use code annotations to
automatically generate a users guide by doing
\href{http://www-cs-faculty.stanford.edu/~uno/lp.html}{\emph{literate
programming}} {[}2{]}.

\textbf{Pitfalls:} Code projects that
have released code without adequate documentation have faced many
challenges.

\begin{enumerate}
\def\labelenumi{(\arabic{enumi})}
\item
  \begin{quote}
  Users tried the software and then dropped it from frustration.
  \end{quote}
\item
  \begin{quote}
  The software was too useful to drop, so the developers ended up
  fielding innumerable queries from users.
  \end{quote}
\item
  \begin{quote}
  Users did not understand the limitations of the software and obtained
  wrong results, which ended up casting doubt on the quality of
  software.
  \end{quote}
\item
  \begin{quote}
  Documentation was not maintained and hence became incompatible with
  the actual software over time, thereby misleading users.
  \end{quote}
\item
  \begin{quote}
  Better, more easily usable software came along, and users switched to
  the other software.
  \end{quote}
\end{enumerate}

Many of these challenges can lead to software becoming obsolete, and as
a result the developers may even lose their funding.

\textbf{References}

{[}1{]}
\href{https://en.wikipedia.org/wiki/Software_documentation}{\emph{https://en.wikipedia.org/wiki/Software\_documentation}}

{[}2{]}
\href{http://www-cs-faculty.stanford.edu/~uno/lp.html}{\emph{http://www-cs-faculty.stanford.edu/\textasciitilde{}uno/lp.html}}

This document was prepared by Anshu Dubey with key contributions from
Roscoe A. Bartlett, Ulrike Yang, and Ethan Coon.

\end{document}
