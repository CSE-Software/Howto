\documentclass[]{article}
% Added package start here
\usepackage{array}
\usepackage{geometry} % for margin of paper
\geometry{margin=1.2in}
\usepackage{fancyhdr} % for headers and footers
\usepackage{graphicx} % for image
\graphicspath{ {media/} }
\pagenumbering{gobble} % Delete page number
% End here
\usepackage{lmodern}
\usepackage{amssymb,amsmath}
\usepackage{ifxetex,ifluatex}
\usepackage{fixltx2e} % provides \textsubscript
\ifnum 0\ifxetex 1\fi\ifluatex 1\fi=0 % if pdftex
  \usepackage[T1]{fontenc}
  \usepackage[utf8]{inputenc}
\else % if luatex or xelatex
  \ifxetex
    \usepackage{mathspec}
  \else
    \usepackage{fontspec}
  \fi
  \defaultfontfeatures{Ligatures=TeX,Scale=MatchLowercase}
\fi
% use upquote if available, for straight quotes in verbatim environments
\IfFileExists{upquote.sty}{\usepackage{upquote}}{}
% use microtype if available
\IfFileExists{microtype.sty}{%
\usepackage{microtype}
\UseMicrotypeSet[protrusion]{basicmath} % disable protrusion for tt fonts
}{}
\usepackage{hyperref}
\hypersetup{unicode=true,
            pdfborder={0 0 0},
            breaklinks=true}
\urlstyle{same}  % don't use monospace font for urls
\IfFileExists{parskip.sty}{%
\usepackage{parskip}
}{% else
\setlength{\parindent}{0pt}
\setlength{\parskip}{6pt plus 2pt minus 1pt}
}
\setlength{\emergencystretch}{3em}  % prevent overfull lines
\providecommand{\tightlist}{%
  \setlength{\itemsep}{0pt}\setlength{\parskip}{0pt}}
\setcounter{secnumdepth}{0}
% Redefines (sub)paragraphs to behave more like sections
\ifx\paragraph\undefined\else
\let\oldparagraph\paragraph
\renewcommand{\paragraph}[1]{\oldparagraph{#1}\mbox{}}
\fi
\ifx\subparagraph\undefined\else
\let\oldsubparagraph\subparagraph
\renewcommand{\subparagraph}[1]{\oldsubparagraph{#1}\mbox{}}
\fi

\date{}

\begin{document}
% Added information start here
\pagestyle{fancy}
\renewcommand{\headrulewidth}{0pt}
% footer
  \fancyfoot[L]{\footnotesize This material is based upon work supported by the U.S. Department of Energy Office of Science, Advanced Scientific Computing Research and Biological and Environmental Research programs. \begin{flushright} \footnotesize Version 0.2, April 25, 2016 \end{flushright}}
  
\fancypagestyle{empty}{
% header
\fancyhead[C]{\LARGE {What is Software Configuration}\\ \normalsize {The IDEAS Scientific Software Productivity Project} \\ \small {\href{https://ideas-productivity.org/resources/howtos/}{\emph{ideas-productivity.org/resources/howtos/}}}} 
  \fancyhead[R]{\includegraphics[width = 1.8 cm, height = 1.2 cm]{ideas}}	
}
\thispagestyle{empty}
\textbf{\newline}
\textbf{\newline}
\textbf{\newline}
% Added information ends here.

\textbf{Motivation:} Installing scientific libraries or applications
from source requires a system for setting up (configuring) the package
to compile and link the code according to the user's specific platform
and needs. This document introduces three of the most common approaches
used by scientific libraries and applications.

\textbf{Method 1: Makefile Options File:} The simplest way to
communicate the options and machine parameters when building a library
or application is to have the person installing the software
(henceforth, the installer) directly edit a text file that will be read
by the compilation scripts. For example, the installer may edit a file
``Make.Incl'' (or perhaps even Makefile itself) in the source directory
to set the location of compilers and other information needed, such as
the location of BLAS and LAPACK libraries. Often this file is read
directly by the package's make system and uses standard makefile syntax.
An example file may look like the following.\\
\# Contents of Make.Incl file\\
CC=gcc\\
CFLAGS=-L/usr/local/lib -llapack -lblas\\
SHARED=0\\
The advantages of this method are the following.

\begin{itemize}
\item
  \begin{quote}
  For small projects it is simple for the developer to maintain.
  \end{quote}
\item
  \begin{quote}
  It can be used on prototype systems that have a minimal software stack
  since it only depends on make.
  \end{quote}
\end{itemize}

The disadvantages are the following.

\begin{itemize}
\item
  \begin{quote}
  There is no separate configuration step prior to compiling the package
  during which the options provided by the installer can be tested.
  \end{quote}
\item
  \begin{quote}
  Thus, compilations will typically be interrupted, as the installer
  likely will need to repeatedly fix the options that are provided.
  \end{quote}
\item
  \begin{quote}
  Any errors will be generated by the compiler, which has no knowledge
  of the meaning of the configuration options. These errors will create
  error messages that are difficult for the installer to trace back to a
  specific incorrect configuration option.
  \end{quote}
\item
  \begin{quote}
  If a package requires specific information about the system, such as
  the existence of certain mathematical functions or include files, it
  is necessary for the installer to determine the correct values to set
  and manually provide the information.
  \end{quote}
\end{itemize}

A more progressive method for setting configuration options is through a
script that collects from the user (for example, from command line
options and environmental variables) and from the system the information
necessary to build the library, tests the information to make sure that
it is valid, and then utilizes the information to compile and link the
software. Such scripts can be more powerful than having installers edit
a file, but they require more upfront effort to write and require
learning a new scripting language.

\textbf{Method 2: GNU
\href{https://en.wikipedia.org/wiki/GNU_build_system}{\emph{Autotools}}
(}a.k.a. configure\textbf{)} is the most commonly used configuration
system. The installer enters the following.

\begin{quote}
./configure {[}-\/-prefix{]} {[}options{]}

make

make install
\end{quote}
GNU Autotools includes Autoconf for generating the configure script and
Automake and Libtool used by the package developer to simplify providing
makefiles and building libraries for the package.

\textbf{Method 3: \href{http://cmake.org}{\emph{CMake}}} is a more
recent alternative to GNU Autotools utilized by some scientific and
mathematical libraries. The installer enters the following.

\begin{quote}
./cmake {[}options{]}

make

make install
\end{quote}
One CMake feature (which GNU Autotools does not have) is that it can
generate all the data files needed for IDE systems such as
\href{http://eclipse}{\emph{Eclipse}} and Microsoft
\href{https://www.visualstudio.com/}{\emph{Visual Studio}}.

Several common difficulties occur with GNU Autotools and CMake.

\begin{itemize}
\item
  \begin{quote}
  Software developers need to learn an entirely new language syntax,
  \href{http://pubs.opengroup.org/onlinepubs/9699919799/utilities/m4.html}{\emph{M4}}
  or
  \href{https://cmake.org/cmake/help/v3.0/manual/cmake-language.7.html\#syntax}{\emph{CMake}}.
  Thus, this script commonly is pieced together from other projects,
  contains errors, and can be difficult to maintain.
  \end{quote}
\item
  \begin{quote}
  If the configure or CMake fails, debugging is difficult even for
  experts.
  \end{quote}
\end{itemize}

Advantages of the CMake and GNU Autotools systems over a basic makefile
system include:

\begin{itemize}
\item
  \begin{quote}
  Abstracting software installation details from users.
  \end{quote}
\item
  \begin{quote}
  Determining many machine parameters automatically and providing
  conventions for setting standard options.
  \end{quote}
\item
  \begin{quote}
  Automating many parts of the dependency finding and testing process.
  \end{quote}
\item
  \begin{quote}
  Simplifying the generation of shared libraries.
  \end{quote}
\item
  \begin{quote}
  Simplifying the management of large complex projects with many source
  directories and dependencies.
  \end{quote}
\end{itemize}

Other partial configuration systems include the use of third-party
utilities to keep track of what libraries have been installed and what
options they used (for example,
\href{http://www.freedesktop.org/wiki/Software/pkg-config/}{\emph{pkg-config}},
or \href{https://docs.python.org/install/}{\emph{setup.py}}).

This document was prepared by Jason Sarich with key contributions from
Roscoe Bartlett, Todd Gamblin and Barry Smith.

\end{document}
