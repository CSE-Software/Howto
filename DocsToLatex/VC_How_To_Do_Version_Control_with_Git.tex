\documentclass[]{article}
% Added package start here
\usepackage{array}
\usepackage{geometry} % for margin of paper
\geometry{margin=1.2in, bottom=1.5in}
\usepackage{fancyhdr} % for headers and footers
\usepackage{graphicx} % for image
\graphicspath{ {media/} }
\pagenumbering{gobble} % Delete page number
% End here
\usepackage{lmodern}
\usepackage{amssymb,amsmath}
\usepackage{ifxetex,ifluatex}
\usepackage{fixltx2e} % provides \textsubscript
\ifnum 0\ifxetex 1\fi\ifluatex 1\fi=0 % if pdftex
  \usepackage[T1]{fontenc}
  \usepackage[utf8]{inputenc}
\else % if luatex or xelatex
  \ifxetex
    \usepackage{mathspec}
  \else
    \usepackage{fontspec}
  \fi
  \defaultfontfeatures{Ligatures=TeX,Scale=MatchLowercase}
\fi
% use upquote if available, for straight quotes in verbatim environments
\IfFileExists{upquote.sty}{\usepackage{upquote}}{}
% use microtype if available
\IfFileExists{microtype.sty}{%
\usepackage{microtype}
\UseMicrotypeSet[protrusion]{basicmath} % disable protrusion for tt fonts
}{}
\usepackage{hyperref}
\hypersetup{unicode=true,
            pdfborder={0 0 0},
            breaklinks=true}
\urlstyle{same}  % don't use monospace font for urls
\IfFileExists{parskip.sty}{%
\usepackage{parskip}
}{% else
\setlength{\parindent}{0pt}
\setlength{\parskip}{6pt plus 2pt minus 1pt}
}
\setlength{\emergencystretch}{3em}  % prevent overfull lines
\providecommand{\tightlist}{%
  \setlength{\itemsep}{0pt}\setlength{\parskip}{0pt}}
\setcounter{secnumdepth}{0}
% Redefines (sub)paragraphs to behave more like sections
\ifx\paragraph\undefined\else
\let\oldparagraph\paragraph
\renewcommand{\paragraph}[1]{\oldparagraph{#1}\mbox{}}
\fi
\ifx\subparagraph\undefined\else
\let\oldsubparagraph\subparagraph
\renewcommand{\subparagraph}[1]{\oldsubparagraph{#1}\mbox{}}
\fi

\date{}

\begin{document}
% Added information start here
\pagestyle{fancy}
\renewcommand{\headrulewidth}{0pt}
% footer
  \fancyfoot[L]{\footnotesize This material is based upon work supported by the U.S. Department of Energy Office of Science, Advanced Scientific Computing Research and Biological and Environmental Research programs. \begin{flushright} \footnotesize Version 0.2, April 25, 2016 \end{flushright}}
  
\fancypagestyle{empty}{
% header
\fancyhead[C]{\LARGE {How to Do Version Control with Git}\\ \LARGE {in your CSE Software Project}\\ \normalsize {The IDEAS Scientific Software Productivity Project} \\ \small {\href{https://ideas-productivity.org/resources/howtos/}{\emph{ideas-productivity.org/resources/howtos/}}}} 
  \fancyhead[R]{\includegraphics[width = 1.8 cm, height = 1.2 cm]{ideas}}	
}
\thispagestyle{empty}
\textbf{\newline}
\textbf{\newline}
\textbf{\newline}
% Added information ends here.

\textbf{Overview:} The distributed version control system Git can be
used to establish effective development and integration processes.
Achieving this objective requires a basic understanding of Git usage and
development workflows.

\textbf{Target Audience:} CSE software project leaders and developers
who would like to adopt an appropriate and efficient version control
workflow using Git for their software-intensive projects.

\textbf{Purpose:} Describe the basic setup and usage of Git, and outline
the different basic building blocks for constructing effective workflows
for single software source Git repositories.

\textbf{Prerequisites:} First read the document
\href{https://docs.google.com/document/d/1LHT4e-BjB31BcCSL42xSI5GBNCNpQ-SS5K5iyStH6sw}{\emph{What
Is Version Control}}.

\textbf{Basic Git Setup:} Before using Git on a new machine, perform the
following minimal setup:

\begin{itemize}
\item
  \begin{quote}
  Set up minimal Git settings for your account, including ``user.name,''
  ``user.email,'' ``color.ui,'' ``push.default,'' and ``rerere.enabled''
  {[}\protect\hyperlink{h.f53m6pxt4rgc}{\emph{1}}{]}.
  \end{quote}
\item
  \begin{quote}
  Install scripts locally for the Git shell prompt
  (\href{https://raw.github.com/git/git/master/contrib/completion/git-prompt.sh}{\emph{git-prompt.sh}})
  and Git tab completion
  (\href{https://raw.github.com/git/git/master/contrib/completion/git-completion.bash}{\emph{git-completion.bash}}),
  and add them to your shell (see documentation in the scripts).
  \end{quote}
\end{itemize}

\textbf{Learn to Use Git:} Understanding Git from an algorithms and
data-structure perspective, rather than just learning commands, can
increase software quality and developer productivity.

\begin{itemize}
\item
  \begin{quote}
  If you are a self-learner, review the \emph{\textbf{Git Tutorial and
  Reference Collection}}
  {[}\protect\hyperlink{id.dpuheqwt966g}{\emph{2}}{]}.
  \end{quote}
\item
  \begin{quote}
  For a more structured approach, take the course \emph{\textbf{How to
  Use Git and GitHub}}
  {[}\protect\hyperlink{id.91we93wr27j3}{\emph{3}}{]}.
  \end{quote}
\item
  \begin{quote}
  Search Google for specific issues. StackOverflow often has an exact
  solution.
  \end{quote}
\end{itemize}

\textbf{Basic Tips for Using Git:} The following basic guidelines and
tips apply to all Git workflows
{[}\protect\hyperlink{h.f53m6pxt4rgc}{\emph{1}}{]}.

\begin{itemize}
\item
  \begin{quote}
  Format commit messages using a 50-char (or so) summary line, followed
  by a blank newline, then (optionally) longer explanatory text in
  paragraphs up to 72 chars wide
  {[}\protect\hyperlink{id.duiqy3srokip}{\emph{6}}{]}.
  \end{quote}
\item
  \begin{quote}
  Create logical commits (see ``SEPARATE CHANGES'' in
  ``gitworkflows(7)''
  {[}\protect\hyperlink{id.m6th6vxlbrrd}{\emph{5}}{]} and ``One Commit
  per Logical Change Solution'' in the Udacity Git course
  {[}\protect\hyperlink{id.91we93wr27j3}{\emph{3}}{]}).
  \end{quote}
\item
  \begin{quote}
  Create local commits to local branch(es) before using commands that
  might pollute or destroy uncommitted changes (e.g., ``git pull,''
  ``git checkout,'' ``git reset,'' ``git rebase'').
  \end{quote}
\item
  \begin{quote}
  Back up your local branch after every few hours of work to some remote
  Git repo.
  \end{quote}
\item
  \begin{quote}
  Use ``git reflog,'' ``git checkout,'' ``git reset -\/-hard,'' or a
  similar command to recover an earlier state of your local repository.
  Previous states can almost always be restored.
  \end{quote}
\item
  \begin{quote}
  Don't commit large (generated) binary files to a Git repository.
  \href{https://git-lfs.github.com}{\emph{Git LFS}} may help.
  \end{quote}
\item
  \begin{quote}
  Never force push to a remote shared branch using ``git push -f'''
  unless you and everyone else sharing the branch know what this means.
  Know how to
  \href{https://github.com/blog/2051-protected-branches-and-required-status-checks}{\emph{protect
  branches}} in your git hosting system of choice.
  \end{quote}
\item
  \begin{quote}
  Create local ``checkpoint'' commits and then cleanup commits with
  ``git rebase -i @\{u\}'' before pushing to a remote shared branch (be
  careful not to rebase public commits).
  \end{quote}
\end{itemize}

\textbf{Git Workflow Building Blocks:} When choosing or constructing a
Git-based workflow, start with the simplest workflow that meets the
project's needs and is appropriate to the level of current Git knowledge
and skill of the developers. Then, as the project is presented with more
challenges, consider augmenting the workflow using the following
workflow building blocks (steps 2-5 can be added in any order)
{[}\protect\hyperlink{id.bol9ubml03th}{\emph{4}}{]}:

\begin{enumerate}
\def\labelenumi{\arabic{enumi}.}
\item
  \begin{quote}
  \textbf{Start: The Simple Centralized Continuous Integration (CI)
  Workflow} has all developers pull from and push to the shared
  ``master'' branch in the one shared repository `origin' (i.e., the
  basic SVN workflow). This is a simple but effective agile-consistent
  workflow and is a good choice for many simpler projects.
  \end{quote}
\item
  \begin{quote}
  \textbf{Add a ``develop'' branch} in order to provide a more stable
  ``master'' branch that is updated on a regular, frequent basis.
  \end{quote}
\item
  \begin{quote}
  \textbf{Add shorter-lived topic branches} for sets of related commits
  (e.g., refactors, bug fixes, work on features) to facilitate easy
  collaborating, code reviews, and back-outs of many commits (which all
  improve stability of the main development branch).
  \end{quote}
\item
  \begin{quote}
  \textbf{Add release branches} for named, supported (multiple) releases
  of the software and release tags to provide patch releases and full
  traceability of software versions.
  \end{quote}
\item
  \begin{quote}
  \textbf{Add longer-lived feature branches} when appropriate to keep
  the development of some new features isolated on topic branches until
  they are ready to be released to customers or to avoid cluttering the
  Git history in case they never make the cut and are therefore never
  merged into the main development branch. However, long-lived feature
  branches (as opposed to implementing a feature in several
  shorter-lived topic branches) can lead to later risky and expensive
  merges into the main development branch.
  \end{quote}
\item
  \begin{quote}
  \textbf{Add one or more throwaway integration test branches} to test
  the integration of the various topic and feature branches that are not
  yet merged into the main development branch. This procedure helps
  detect integration problems early and makes more effective usage of
  computer testing resources.
  \end{quote}
\item
  \begin{quote}
  \textbf{End: The git.git workflow (i.e., ``gitworkflows(7)'')} is a
  combination of the above workflow building blocks and is used for
  developing many projects, including the Git source code itself (i.e.,
  ``git.git'' {[}\protect\hyperlink{id.m6th6vxlbrrd}{\emph{5}}{]}) and
  the Linux kernel. However, because the git.git workflow is complex and
  labor-intensive, its use is justified only for projects where all the
  developers are Git savvy and the project's challenges justify its
  usage.
  \end{quote}
\end{enumerate}

\protect\hypertarget{h.9qg5mj337055}{}{}\textbf{References}

\protect\hypertarget{h.f53m6pxt4rgc}{}{\protect\hypertarget{id.pmt7367owle1}{}{}}{[}1{]}
Roscoe Bartlett. \emph{Critical Beginner Git Usage Tips}. IDEAS
Scientific Software Productivity Project.
\href{https://ideas-productivity.org/resources/howtos/git-tutorial-and-reference-collection/beginner-tips}{\emph{https://ideas-productivity.org/resources/howtos/git-tutorial-and-reference-collection/beginner-tips}}

\protect\hypertarget{id.dpuheqwt966g}{}{}{[}2{]} Roscoe Bartlett.
\emph{Git Tutorial and Reference Collection}. IDEAS Scientific Software
Productivity Project.
\href{https://ideas-productivity.org/resources/howtos/git-tutorial-and-reference-collection}{\emph{https://ideas-productivity.org/resources/howtos/git-tutorial-and-reference-collection}}

\protect\hypertarget{id.91we93wr27j3}{}{}{[}3{]} Udacity. \emph{How to
Use Git and GitHub}.
\href{https://www.udacity.com/course/how-to-use-git-and-github--ud775}{\emph{https://www.udacity.com/course/how-to-use-git-and-github-\/-ud775}}

\protect\hypertarget{id.bol9ubml03th}{}{}{[}4{]} Roscoe Bartlett.
\emph{Design Patterns for Incrementally Expanding Git Workflows for
Research-Based Projects.} IDEAS Scientific Software Productivity
Project. To be published. \newline
\href{https://docs.google.com/document/d/1uVQYI2cmNx09fDkHDA136yqDTqayhxqfvjFiuUue7wo}{\emph{https://docs.google.com/document/d/1uVQYI2cmNx09fDkHDA136yqDTqayhxqfvjFiuUue7wo}}

\protect\hypertarget{id.m6th6vxlbrrd}{}{}{[}5{]} \emph{gitworkflows(7) -
An overview of recommended workflows with Git}, \newline
\href{https://www.kernel.org/pub/software/scm/git/docs/gitworkflows.html}{\emph{https://www.kernel.org/pub/software/scm/git/docs/gitworkflows.html}}

\protect\hypertarget{id.duiqy3srokip}{}{}{[}6{]} Chris Beams, \emph{How
to Write a Git Commit Message},
\href{http://chris.beams.io/posts/git-commit/}{\emph{http://chris.beams.io/posts/git-commit/}}

This document was prepared by Roscoe A. Bartlett with key contributions
from James M. Willenbring, Michael A. Heroux and Todd Gamblin.

\end{document}
