\documentclass[]{article}
% Added package start here
\usepackage{array}
\usepackage{geometry} % for margin of paper
\geometry{margin=1.2in}
\usepackage{fancyhdr} % for headers and footers
\usepackage{graphicx} % for image
\graphicspath{ {media/} }
\pagenumbering{gobble} % Delete page number
% End here
\usepackage{lmodern}
\usepackage{amssymb,amsmath}
\usepackage{ifxetex,ifluatex}
\usepackage{fixltx2e} % provides \textsubscript
\ifnum 0\ifxetex 1\fi\ifluatex 1\fi=0 % if pdftex
  \usepackage[T1]{fontenc}
  \usepackage[utf8]{inputenc}
\else % if luatex or xelatex
  \ifxetex
    \usepackage{mathspec}
  \else
    \usepackage{fontspec}
  \fi
  \defaultfontfeatures{Ligatures=TeX,Scale=MatchLowercase}
\fi
% use upquote if available, for straight quotes in verbatim environments
\IfFileExists{upquote.sty}{\usepackage{upquote}}{}
% use microtype if available
\IfFileExists{microtype.sty}{%
\usepackage{microtype}
\UseMicrotypeSet[protrusion]{basicmath} % disable protrusion for tt fonts
}{}
\usepackage{hyperref}
\hypersetup{unicode=true,
            pdfborder={0 0 0},
            breaklinks=true}
\urlstyle{same}  % don't use monospace font for urls
\IfFileExists{parskip.sty}{%
\usepackage{parskip}
}{% else
\setlength{\parindent}{0pt}
\setlength{\parskip}{6pt plus 2pt minus 1pt}
}
\setlength{\emergencystretch}{3em}  % prevent overfull lines
\providecommand{\tightlist}{%
  \setlength{\itemsep}{0pt}\setlength{\parskip}{0pt}}
\setcounter{secnumdepth}{0}
% Redefines (sub)paragraphs to behave more like sections
\ifx\paragraph\undefined\else
\let\oldparagraph\paragraph
\renewcommand{\paragraph}[1]{\oldparagraph{#1}\mbox{}}
\fi
\ifx\subparagraph\undefined\else
\let\oldsubparagraph\subparagraph
\renewcommand{\subparagraph}[1]{\oldsubparagraph{#1}\mbox{}}
\fi

\date{}

\begin{document}
% Added information start here
\pagestyle{fancy}
\renewcommand{\headrulewidth}{0pt}
% footer
  \fancyfoot[L]{\footnotesize This material is based upon work supported by the U.S. Department of Energy Office of Science, Advanced Scientific Computing Research and Biological and Environmental Research programs. \begin{flushright} \footnotesize Version 0.2, April 25, 2016 \end{flushright}}
  
\fancypagestyle{empty}{
% header
\fancyhead[C]{\LARGE {\textbf{What is Github}}\\ \normalsize {The IDEAS Scientific Software Productivity Project} \\ \small {\href{https://ideas-productivity.org/resources/howtos/}{\emph{ideas-productivity.org/resources/howtos/}}}} 
  \fancyhead[R]{\includegraphics[width = 3 cm, height = 1.5 cm]{ideas_Whatis}}	
}
\thispagestyle{empty}
\textbf{\newline}
% Added information ends here.

Github (\href{https://github.com/}{\emph{github.org}}) is a service
(website) for accessing and managing git repositories. Github also
provides a variety of tools for the distributed (multi-site) software
development process.

\textbf{Useful Prerequisites:}
\href{https://docs.google.com/document/d/1LHT4e-BjB31BcCSL42xSI5GBNCNpQ-SS5K5iyStH6sw/edit}{\emph{What
is version control?}} and
\href{https://docs.google.com/document/d/1mSrTdzZdLDz-YRquzABNhI87tGd_zP04ctyscmZ_Phw/edit}{\emph{How
to do version control with Git}}.

Github provides a number of features.

\begin{itemize}
\item
  \begin{quote}
  It is free for open development projects (i.e. projects that make
  their repositories readable) and private projects with a small number
  of users or from education institutes. Costs for larger private
  repositories are modest.
  \end{quote}
\item
  \begin{quote}
  It is trivial to add new members to the repositories with varying
  levels of access permissions (but access control options are fairly
  limited compared to other tools).
  \end{quote}
\item
  \begin{quote}
  There is an issues board that can be used for managing bug reports,
  proposed projects etc.
  \end{quote}
\item
  \begin{quote}
  There is a wiki where documentation, coding development requirements
  can be maintained.
  \end{quote}
\item
  \begin{quote}
  It is possible to display graphically both the repositories' history
  and the history of contributions to the repositories.
  \end{quote}
\item
  \begin{quote}
  One can manage repositories on Github using any web browser and does
  not require installing any software on your computer.
  \end{quote}
\item
  \begin{quote}
  There is a system for managing contributions to the repository from
  outside developers based on \emph{pull requests}.
  \end{quote}
\item
  \begin{quote}
  There are hooks to many free services. One example is Travis CI for
  free continuous integration testing.
  \end{quote}
\end{itemize}

\textbf{What are pull requests?} They are methods for developers, both
within the project team and completely outside the team, to write new
code and then request that it be reviewed and tested before being merged
into repository. Github provides a simple web based graphical interface
to submit pull requests, hooks to allow automatically testing (e.g.
using Travis CI) of the proposed code when it is submitted, and a
graphical interface for examining and commenting on the proposed code.
Based on feedback the submitter can rework the code and then update the
pull requests.

\textbf{What to read next:}
\href{https://docs.google.com/document/d/1sfVMcCNwJIoUpVSyJ6HEmi3Dx3NhCpR_7BOz5cfhnAg/edit}{\emph{How
to use Github}}

This document was prepared by Barry Smith and XXX with key contributions
from YYY.

This material is based upon work supported by the U.S. Department of
Energy Office of Science, Advanced Scientific Computing Research and
Biological and Environmental Research programs.

\end{document}
