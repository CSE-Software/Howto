\documentclass[]{article}
% Added package start here
\usepackage{array}
\usepackage{geometry} % for margin of paper
\geometry{margin=1.2in}
\usepackage{fancyhdr} % for headers and footers
\usepackage{graphicx} % for image
\graphicspath{ {media/} }
\pagenumbering{gobble} % Delete page number
% End here
\usepackage{lmodern}
\usepackage{amssymb,amsmath}
\usepackage{ifxetex,ifluatex}
\usepackage{fixltx2e} % provides \textsubscript
\ifnum 0\ifxetex 1\fi\ifluatex 1\fi=0 % if pdftex
  \usepackage[T1]{fontenc}
  \usepackage[utf8]{inputenc}
\else % if luatex or xelatex
  \ifxetex
    \usepackage{mathspec}
  \else
    \usepackage{fontspec}
  \fi
  \defaultfontfeatures{Ligatures=TeX,Scale=MatchLowercase}
\fi
% use upquote if available, for straight quotes in verbatim environments
\IfFileExists{upquote.sty}{\usepackage{upquote}}{}
% use microtype if available
\IfFileExists{microtype.sty}{%
\usepackage{microtype}
\UseMicrotypeSet[protrusion]{basicmath} % disable protrusion for tt fonts
}{}
\usepackage{hyperref}
\hypersetup{unicode=true,
            pdfborder={0 0 0},
            breaklinks=true}
\urlstyle{same}  % don't use monospace font for urls
\IfFileExists{parskip.sty}{%
\usepackage{parskip}
}{% else
\setlength{\parindent}{0pt}
\setlength{\parskip}{6pt plus 2pt minus 1pt}
}
\setlength{\emergencystretch}{3em}  % prevent overfull lines
\providecommand{\tightlist}{%
  \setlength{\itemsep}{0pt}\setlength{\parskip}{0pt}}
\setcounter{secnumdepth}{0}
% Redefines (sub)paragraphs to behave more like sections
\ifx\paragraph\undefined\else
\let\oldparagraph\paragraph
\renewcommand{\paragraph}[1]{\oldparagraph{#1}\mbox{}}
\fi
\ifx\subparagraph\undefined\else
\let\oldsubparagraph\subparagraph
\renewcommand{\subparagraph}[1]{\oldsubparagraph{#1}\mbox{}}
\fi

\date{}

\begin{document}
% Added information start here
\pagestyle{fancy}
\renewcommand{\headrulewidth}{0pt}
% footer
  \fancyfoot[L]{\footnotesize This material is based upon work supported by the U.S. Department of Energy Office of Science, Advanced Scientific Computing Research and Biological and Environmental Research programs. \begin{flushright} \footnotesize Version 0.2, April 25, 2016 \end{flushright}}
  
\fancypagestyle{empty}{
% header
\fancyhead[C]{\LARGE {How to use Github}\\ \normalsize {The IDEAS Scientific Software Productivity Project} \\ \small {\href{https://ideas-productivity.org/resources/howtos/}{\emph{ideas-productivity.org/resources/howtos/}}}} 
  \fancyhead[R]{\includegraphics[width = 1.8 cm, height = 1.2 cm]{ideas}}	
}
\thispagestyle{empty}
\textbf{\newline}
\textbf{\newline}
\textbf{\newline}
% Added information ends here.

\textbf{Overview:} This document describes best practices for using
GitHub as a collaborative tool for developing scientific software.
Because of the enthusiasm of the software development community for
GitHub, there has been a lot of activity in this area, and we do not
attempt to comprehensively catalog every approach here. Rather, we
intend to draw attention to some workflows and rules of thumb to help
scientific programmers to be more productive.

\textbf{Target Audience:} Scientific software project leaders and
developers.

\textbf{Prerequisites:} First read the documents
\href{https://docs.google.com/document/d/1LHT4e-BjB31BcCSL42xSI5GBNCNpQ-SS5K5iyStH6sw/edit}{\emph{What
is version control?}},
\href{https://docs.google.com/document/d/1mSrTdzZdLDz-YRquzABNhI87tGd_zP04ctyscmZ_Phw/edit}{\emph{How
to do version control with Git}}, and
\href{https://docs.google.com/document/d/1XoLBPwQ0lolYvEUsBpVcsVv3jUx6X9g4LL_x0xCsHsA/edit}{\emph{What
is Github?}}

\textbf{Resources:}

\begin{itemize}
\item
  \begin{quote}
  \href{https://www.atlassian.com/git/tutorials/comparing-workflows}{\emph{Comparing
  Git workflows}}
  \end{quote}
\item
  \begin{quote}
  \href{http://nvie.com/posts/a-successful-git-branching-model/}{\emph{A
  successful Git branching model}}
  \end{quote}
\item
  \begin{quote}
  \href{https://help.github.com/}{\emph{GitHub's help site}}
  \end{quote}
\end{itemize}

This document was prepared by Barry Smith and Jeffrey Johnson with key
contributions from YYY.

This material is based upon work supported by the U.S. Department of
Energy Office of Science, Advanced Scientific Computing Research and
Biological and Environmental Research programs.

\end{document}
