\documentclass[]{article}
% Added package start here
\usepackage{array}
\usepackage{geometry} % for margin of paper
\geometry{margin=1.2in}
\usepackage{fancyhdr} % for headers and footers
\usepackage{graphicx} % for image
\graphicspath{ {media/} }
\pagenumbering{gobble} % Delete page number
% End here
\usepackage{lmodern}
\usepackage{amssymb,amsmath}
\usepackage{ifxetex,ifluatex}
\usepackage{fixltx2e} % provides \textsubscript
\ifnum 0\ifxetex 1\fi\ifluatex 1\fi=0 % if pdftex
  \usepackage[T1]{fontenc}
  \usepackage[utf8]{inputenc}
\else % if luatex or xelatex
  \ifxetex
    \usepackage{mathspec}
  \else
    \usepackage{fontspec}
  \fi
  \defaultfontfeatures{Ligatures=TeX,Scale=MatchLowercase}
\fi
% use upquote if available, for straight quotes in verbatim environments
\IfFileExists{upquote.sty}{\usepackage{upquote}}{}
% use microtype if available
\IfFileExists{microtype.sty}{%
\usepackage{microtype}
\UseMicrotypeSet[protrusion]{basicmath} % disable protrusion for tt fonts
}{}
\usepackage{hyperref}
\hypersetup{unicode=true,
            pdfborder={0 0 0},
            breaklinks=true}
\urlstyle{same}  % don't use monospace font for urls
\IfFileExists{parskip.sty}{%
\usepackage{parskip}
}{% else
\setlength{\parindent}{0pt}
\setlength{\parskip}{6pt plus 2pt minus 1pt}
}
\setlength{\emergencystretch}{3em}  % prevent overfull lines
\providecommand{\tightlist}{%
  \setlength{\itemsep}{0pt}\setlength{\parskip}{0pt}}
\setcounter{secnumdepth}{0}
% Redefines (sub)paragraphs to behave more like sections
\ifx\paragraph\undefined\else
\let\oldparagraph\paragraph
\renewcommand{\paragraph}[1]{\oldparagraph{#1}\mbox{}}
\fi
\ifx\subparagraph\undefined\else
\let\oldsubparagraph\subparagraph
\renewcommand{\subparagraph}[1]{\oldsubparagraph{#1}\mbox{}}
\fi

\date{}

\begin{document}
% Added information start here
\pagestyle{fancy}
\renewcommand{\headrulewidth}{0pt}
% footer
  \fancyfoot[L]{\footnotesize This material is based upon work supported by the U.S. Department of Energy Office of Science, Advanced Scientific Computing Research and Biological and Environmental Research programs. \begin{flushright} \footnotesize Version 0.2, April 25, 2016 \end{flushright}}
  
\fancypagestyle{empty}{
% header
\fancyhead[C]{\LARGE {What is Version Control}\\ \normalsize {The IDEAS Scientific Software Productivity Project} \\ \small {\href{https://ideas-productivity.org/resources/howtos/}{\emph{ideas-productivity.org/resources/howtos/}}}} 
  \fancyhead[R]{\includegraphics[width = 1.8 cm, height = 1.2 cm]{ideas}}	
}
\thispagestyle{empty}
\textbf{\newline}
\textbf{\newline}
\textbf{\newline}
% Added information ends here.

\textbf{Motivation:} Software source, documentation, and other important
(text) documents should be managed with a \textbf{Version Control System
(VCS)} in order to do the following:

\begin{itemize}
\item
  \begin{quote}
  Support safe incremental development (i.e., ``undos'')
  \end{quote}
\item
  \begin{quote}
  Support collaboration among different developers, contributors, and
  customers
  \end{quote}
\item
  \begin{quote}
  Provide traceability from requirements to file changes
  \end{quote}
\item
  \begin{quote}
  Streamline development and testing processes
  \end{quote}
\item
  \begin{quote}
  Provide reproducibility of past results
  \end{quote}
\end{itemize}

Users interact with a VCS through a formal process or \textbf{workflow}.
In this document, we introduce some concepts and terminology of version
control \protect\hyperlink{h.zeieakqm2nff}{\emph{{[}1{]}}}, mention some
of its benefits, describe use cases where it has been helpful (even
critical), and outline some of the major VCS tools
\protect\hyperlink{h.f53m6pxt4rgc}{\emph{{[}2{]}}}.

\textbf{Version Control Definitions and Terminology}

Important terms and concepts in version control (see
\protect\hyperlink{h.zeieakqm2nff}{\emph{{[}1{]}}}) include the
following.

\begin{itemize}
\item
  \begin{quote}
  In a \textbf{Centralized VCS,} there is a single \textbf{repository},
  and the user \textbf{workflow} consists of \emph{\textbf{checking
  out}} stored versions, updating the checked-out copy, and
  \emph{\textbf{checking in,} or \textbf{committing}} updated versions
  to the central repository. \emph{Versions} of software are referred to
  as \emph{\textbf{revisions}} or \emph{\textbf{commits}}. Users check
  out a single version at a time, and only the central repository stores
  all commits. If the central repository is lost without backups, the
  repository's entire history will be lost. Most VCS tools allow
  concurrent editing of the same files, requiring different versions to
  be merged. This may result in \textbf{merge conflicts}, which must be
  resolved manually before updating the central repository. (Note:
  merging and resolving merge conflicts can be risky and time
  consuming.)
  \end{quote}
\item
  \begin{quote}
  A \textbf{Distributed VCS} allows multiple complete copies of the same
  repository, and changes are moved back and forth between different
  repositories using various processes and workflows. Distributed VCS
  enables workflows that are well suited for build \& test, code review,
  collaboration, and concurrent \textbf{branches}.
  \end{quote}
\item
  \begin{quote}
  A \textbf{branch} is an ordered set of commits representing a single
  history of changes to the files in a repository. Most systems support
  the creation and merging of branches. Branches are an especially
  important concept and tool in distributed VCS tools and processes.
  \end{quote}
\end{itemize}

\textbf{Version Control Tools}

Many high-quality open-source and commercial VCS are available
\protect\hyperlink{h.f53m6pxt4rgc}{\emph{{[}2{]}}}. Some of the more
popular free open-source version control tools are the following:

\begin{itemize}
\item
  \begin{quote}
  \textbf{Subversion (SVN)} is a popular centralized VCS started in
  2000. The user interface is fairly simple and easy to learn. This is
  largely due to the simplicity of the centralized VC workflow, but
  branches are also supported.
  \end{quote}
\item
  \begin{quote}
  \textbf{Git} is a more recent distributed VCS started in 2005 to
  support the development of the Linux kernel. In recent years Git has
  become the most popular and dominant VCS in use, due to collaborative
  workflows enabled by distributed VCS and made popular by the
  \href{http://www.github.com}{\emph{GitHub}} code hosting site. Git has
  a large and complex user interface, and it takes significant effort to
  learn. Git is poorly suited for managing large binary data files (but
  extensions like \href{https://git-lfs.github.com}{\emph{Git Large File
  Storage (LFS)}} may help). Nevertheless, many large and complex
  projects use Git in all sectors of software development (but often use
  many smaller Git repositories rather than fewer large Git
  repositories).
  \end{quote}
\item
  \begin{quote}
  \textbf{Mercurial} (hg), another distributed VCS, started around 2005.
  It is less popular than Git but is generally considered to have a
  simpler user interface. Mercurial is used by many projects (e.g., it
  is the primary VCS tool for Facebook).
  \end{quote}
\item
  \begin{quote}
  \textbf{CVS} was the first popular, successful open-source VCS. It is
  an older centralized VCS, and Subversion was created as its successor.
  Although new projects no longer choose CVS, its influence and legacy
  are important to note when considering VCS tools and workflows.
  \end{quote}
\end{itemize}

Several tools exist to convert and interoperate between different
open-source VCS (e.g.,
\href{https://www.kernel.org/pub/software/scm/git/docs/v1.5.0/git-svn.html}{\emph{git-svn}},
\href{https://www.mercurial-scm.org/wiki/HgGit}{\emph{hg-git}}).
However, these tools vary greatly in maturity, performance, scalability,
and usability.

\textbf{Use Cases for Version Control}

Version control is useful in numerous situations. Because most VCS tools
are oriented to text file lines, users can manage and collaborate on
nearly any set of text files in a VCS. The following are examples.

\begin{itemize}
\item
  \begin{quote}
  \textbf{Software development by a single individual:} Enables the
  developer to keep track of older versions, support reproducibility of
  past results, pursue incremental commits with undo, simplify automated
  testing, help with porting, etc.
  \end{quote}
\item
  \begin{quote}
  \textbf{Software development by a team:} Aids in collaboration among
  developers, supports code reviews, provides traceability of
  requirements to code changes, etc.
  \end{quote}
\item
  \begin{quote}
  \textbf{Collaborative document writing:} For documents in plain text
  source (e.g., Latex, reStructuredText, Markdown, HTML), allows
  concurrent editing with merges (line by line), tracks who contributed
  to what sections (e.g., using ``blame''), etc.
  \end{quote}
\end{itemize}

\protect\hypertarget{h.9qg5mj337055}{}{}\textbf{References:}

\protect\hypertarget{h.zeieakqm2nff}{}{\protect\hypertarget{id.hbymvf6o75jy}{}{}}{[}1{]}
\href{https://en.wikipedia.org/wiki/Version_control}{\emph{https://en.wikipedia.org/wiki/Version\_control}}

\protect\hypertarget{h.f53m6pxt4rgc}{}{\protect\hypertarget{id.701ggrsblayp}{}{}}{[}2{]}
\href{https://en.wikipedia.org/wiki/List_of_version_control_software}{\emph{https://en.wikipedia.org/wiki/List\_of\_version\_control\_software}}

This document was prepared by Roscoe A. Bartlett with key contributions
from Jim Willenbring and Todd Gamblin.

\end{document}
