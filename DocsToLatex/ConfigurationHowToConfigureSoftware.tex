\documentclass[]{article}
% Added package start here
\usepackage{array}
\usepackage{geometry} % for margin of paper
\geometry{margin=1.2in}
\usepackage{fancyhdr} % for headers and footers
\usepackage{graphicx} % for image
\graphicspath{ {media/} }
\pagenumbering{gobble} % Delete page number
% End here
\usepackage{lmodern}
\usepackage{amssymb,amsmath}
\usepackage{ifxetex,ifluatex}
\usepackage{fixltx2e} % provides \textsubscript
\ifnum 0\ifxetex 1\fi\ifluatex 1\fi=0 % if pdftex
  \usepackage[T1]{fontenc}
  \usepackage[utf8]{inputenc}
\else % if luatex or xelatex
  \ifxetex
    \usepackage{mathspec}
  \else
    \usepackage{fontspec}
  \fi
  \defaultfontfeatures{Ligatures=TeX,Scale=MatchLowercase}
\fi
% use upquote if available, for straight quotes in verbatim environments
\IfFileExists{upquote.sty}{\usepackage{upquote}}{}
% use microtype if available
\IfFileExists{microtype.sty}{%
\usepackage{microtype}
\UseMicrotypeSet[protrusion]{basicmath} % disable protrusion for tt fonts
}{}
\usepackage{hyperref}
\hypersetup{unicode=true,
            pdfborder={0 0 0},
            breaklinks=true}
\urlstyle{same}  % don't use monospace font for urls
\usepackage{longtable, booktabs}
\IfFileExists{parskip.sty}{%
\usepackage{parskip}
}{% else
\setlength{\parindent}{0pt}
\setlength{\parskip}{6pt plus 2pt minus 1pt}
}
\setlength{\emergencystretch}{3em}  % prevent overfull lines
\providecommand{\tightlist}{%
  \setlength{\itemsep}{0pt}\setlength{\parskip}{0pt}}
\setcounter{secnumdepth}{0}
% Redefines (sub)paragraphs to behave more like sections
\ifx\paragraph\undefined\else
\let\oldparagraph\paragraph
\renewcommand{\paragraph}[1]{\oldparagraph{#1}\mbox{}}
\fi
\ifx\subparagraph\undefined\else
\let\oldsubparagraph\subparagraph
\renewcommand{\subparagraph}[1]{\oldsubparagraph{#1}\mbox{}}
\fi

\date{}

\begin{document}
% Added information start here
\pagestyle{fancy}
\renewcommand{\headrulewidth}{0pt}
% footer
  \fancyfoot[L]{\footnotesize This material is based upon work supported by the U.S. Department of Energy Office of Science, Advanced Scientific Computing Research and Biological and Environmental Research programs. \begin{flushright} \footnotesize Version 0.2, April 25, 2016 \end{flushright}}
  
\fancypagestyle{empty}{
% header
\fancyhead[C]{\LARGE {How to Configure Software}\\ \normalsize {The IDEAS Scientific Software Productivity Project} \\ \small {\href{https://ideas-productivity.org/resources/howtos/}{\emph{ideas-productivity.org/resources/howtos/}}}} 
  \fancyhead[R]{\includegraphics[width = 1.8 cm, height = 1.2 cm]{ideas}}	
}
\thispagestyle{empty}
\textbf{\newline}
\textbf{\newline}
\textbf{\newline}
% Added information ends here.

\textbf{Overview:} Most CSE software needs to be installed from source
on a wide variety of machines by end users. Developers of the software
must decide how to enable this installation easily without overburdening
the developers of software or the end users. This document introduces
several approaches to use depending on the contents and scale of the
software package.

\textbf{Target Audience:} Scientific software project leaders and
developers who need to ensure that their software can be installed on a
wide variety of machines.

\textbf{Prerequisites:} First read the document
\href{http://ideas-productivity.org/wordpress/wp-content/uploads/2016/02/IDEAS-WhatIsSoftwareConfiguration-V0.1.pdf}{\emph{What
Is Software Configuration?}}

For simple packages that have almost no dependencies or machine
dependent parameters, the use of (1) an options file is acceptable. For
packages that incrementally build on another package, it can be
reasonable to piggyback on the other package's configuration
information, thus requiring the end user merely to edit a file to
indicate the location of the piggybacked package.

For all other packages we recommend using the open-source tools (2) GNU
\href{https://en.wikipedia.org/wiki/GNU_build_system}{\emph{Autotools}}
or (3) \href{https://cmake.org/}{\emph{CMake}} or (4) rolling your own
configuration system that utilizes the syntax of the Autoconf command
line arguments (see comparison of features in the table below). GNU
Autotools is widely used and has a great deal of support resources on
the web but is a bit idiosyncratic. Almost all Linux users and many Mac
OS users are familiar with and expect the Autotools command line syntax.
CMake is a product of a for-profit company
\href{http://www.kitware.com}{\emph{Kitware}} whose business model is
based on paid customer support (CMake is open-source software and free
community support is available on the open CMake mail lists). CMake
comes with a testing environment CTest (which posts to a web dashboard
CDash) and a packaging system CPack.

\textbf{Creating a GNU Autoconf configure script}

The GNU open source package
\href{http://www.gnu.org/software/autoconf/}{\emph{Autotools}} is
distributed by GNU and is available for any modern system. In brief, the
software developer creates a ``configure.in'' file, which contains all
options that can be set and any tests that should be run, and writes
makefile templates that will be populated with information from the
given options. Next, the developer runs Autoconf to generate a
``configure'' file, which will then be included as part of the source
distribution. Autotools also provides Automake, which helps with the
generation of the makefiles from a Makefile.am file, and Libtool, which
helps to manage building libraries. The details on how to make a usable
``configure.in'' file and makefile templates are beyond the scope of
this document; numerous books and online references can help with these
tasks. For high-performance computing machines that utilize a batch
system -- that is, require submitting all programs to a queue to be run
on the machine -- GNU Autotools can be problematic since it relies on
being able to automatically build and run programs to determine machine
properties.

\textbf{Creating a CMake build}

CMake has very different syntax from the GNU Autoconf configure scripts
but performs essentially the same function. Thus, it has analogous
options and variables that must be communicated to the build system. As
an example, where configure may expect to set the C compiler by checking
the passed in `CC' variable, CMake looks for the option
`CMAKE\_C\_COMPILER' first and if it does not find it, uses the compiler
listed by the environment variable `CC'.

A CMake project must define a CMakeLists.txt file in its distribution.
This file is analogous to the configure.in file for GNU Autoconf builds
(and the Makefile.am used by Automake), in that it describes what
options or variables are expected from the software installer and how
those options will be broadcast to the makefiles that actually compile
the software.

Unlike GNU Autotools, however, CMake provides facilities for use on
high-performance computing machines that utilize a batch system.
Unfortunately, it relies on a database of machine properties, and this
database is often out of date for many high-performance computing sites.
GNU Autotools and CMake both support cross-compiling, which can be used
on a batch-based system, but this disables the useful configure-time
testing that the tools provide.

\textbf{Rolling your own configure script}

A ``configure'' script can also be written without using the GNU
Autotools programs and not requiring the use of M4 for those who want to
avoid it.\footnote{For example, PETSc's configure, written completely in
  Python, provides all the functionality of GNU Autotools as well as the
  ability to install other packages and work on batch computer systems.}
This approach has the advantage of allowing additional features and
behavior not provided by GNU. It is highly recommended that any other
configure script follow the syntax and expected behavior of a GNU
Autoconf-generated script.

\begin{itemize}
\item
  \begin{quote}
  Use the -\/-help option to list and explain the various options
  available.
  \end{quote}
\item
  \begin{quote}
  Use -\/-prefix= to denote where to install header files and binaries.
  \end{quote}
\item
  \begin{quote}
  Check the command line and the environment for compiler variables CC,
  CFLAGS, CXX, CXXFLAGS, and other similar variables. If these are not
  specifically given, then make some guesses based on what executables
  are available.
  \end{quote}
\end{itemize}

No matter how your configure script is created, users expect certain
conventions.

\begin{itemize}
\item
  \begin{quote}
  Use -\/-with-xxx (and -\/-without-xxx) for describing what is
  available in the install environment. For example, if your build
  behaves differently if LAPACK is available or not, then you should use
  a -\/-with-lapack option.\footnote{GNU recommends that you do not use
    locations with these variables but instead add any necessary flags
    to the LDFLAGS environment variable (i.e., GNU prefers
    LDFLAGS=/usr/lib/libpack.a -\/-with-lapack instead of
    -\/-with-lapack-dir=/usr/lib/liblapack.a). We disagree with this
    recommendation because the latter approach allows tests to be
    written for each particular option.}
  \end{quote}
\item
  \begin{quote}
  Use -\/-enable-xxx (and -\/-disable-xxx) to turn on or off features in
  the code, such as options to build fortran interfaces, debugging
  symbols, or shared libraries.
  \end{quote}
\item
  \begin{quote}
  Print out useful messages if there is a problem or inconsistency with
  the options.
  \end{quote}
\end{itemize}

\textbf{Comparison of Software Configuration Features}
\begin{longtable}{ | m{3.5cm} | m{3cm}| m{2.5cm} | m{3cm}| m{3.5cm} |}
\toprule
\textbf{Feature} & \textbf{(1) Parameter file} & \textbf{(2) GNU
Autotools} & \textbf{(3) CMake} & \textbf{(4) Roll your own
configure}\tabularnewline
\midrule
\endhead
Can automatically determine machine parameters & & Yes & Yes & Yes, but
you must provide all these tests as part of your system\tabularnewline\hline
 Requires a large initial investment & & Yes & Yes & Yes, as you are
starting from scratch you cannot leverage previously developed
code\tabularnewline\hline 
Has complex capabilities & & Yes & Yes & Yes, but only if you write
them\tabularnewline\hline
 Can be modified as needed & Yes & Yes & Yes & Yes\tabularnewline
Works with Microsoft Visual Studio & User must manually enter required
information into Visual Studio. This is painful and error prone. & &
Yes, generates native Microsoft Visual Studio project files & Yes, but
you must provide the appropriate code to generate the needed
files\tabularnewline\hline
 Native windows builds (that may then be used from within Microsoft
Visual Studio) & & & Yes, generates native NMake and Ninja build files
for all native Windows compilers & Yes, but you have to write all of
such support by yourself (i.e., can use Makefiles.)\tabularnewline\hline 
Requires learning new scripting language & & Yes & Yes &\tabularnewline\hline 
Can test given options for compatibility & & Yes & Yes & Yes, but you
must provide the tests.\tabularnewline\hline
 Is commonly used, documentation available & & Yes & Yes &\tabularnewline\hline 
Works with IDE's & User must manually enter required information into
the IDE. This is painful and error prone. & & Yes, generates native
project files for XCode, Eclipse, etc. & Yes, but you must provide the
appropriate code to generate the needed files\tabularnewline\hline
 Supports different backend build tools other than Makefiles (e.g. Ninja,
NMake) & User must manually enter required information. This is
generally painful and error prone. & & Yes & Yes, but you must write the
tools to generate such backend build files\tabularnewline\hline
 Portable support for shared libraries & & Yes, but you need to also use
libtool or gmake as well. & Yes & Yes, but you need to provide the
knowledge and logic for every platform\tabularnewline\hline 
Portable automatic generation of dependency information & & Yes, but
requires usage of automake and requires compiler support for generating
dependency info, which they all have & Yes, built into the CMake
executable, not dependent on compiler support & Yes, but you have to
roll your own\tabularnewline\hline 
Provides database for known HPC computer systems & & & Yes, though
sometimes not available for new prototype systems &\tabularnewline\hline
 Includes graphical interface & & & Yes &\tabularnewline
\bottomrule
\end{longtable}
Regardless of the method used to configure, your system should do the
following.

\begin{itemize}
\item
  \begin{quote}
  Provide end users access to the options that were used to configure
  the file and also the internal variables that were set when the
  ``configure'' script ran. This information is helpful for
  reconfiguring a software package to prevent having to duplicate any
  trial-and-error learning process, and it can help keep all
  dependencies consistent for future software builds.
  \end{quote}
\item
  \begin{quote}
  Make proper documentation available for the installer. At the least,
  any required software dependencies should be listed, every
  configuration variable should be explained, and appropriate examples
  should be given. This documentation can be on an installation
  instruction web page or text file included in the distribution, and it
  should also be available from command line help queries.
  \end{quote}
\end{itemize}

\textbf{Common configuration options for scientific software\\
}\\
Many scientific software libraries and applications require the same
information when configuring, such as locations of BLAS and LAPACK. When
a set of interacting software libraries is built, the same options must
be used for each of the libraries. In the past each software package
selected its own name for these options, making installing multiple
packages painful and error prone. The IDEAS team has developed a set of
\href{https://drive.google.com/open?id=18028D6nsuhIrCvJnX6c07r8m_Np4SH-aGXMX4svMs1w}{\emph{standard
configuration options}} that we recommend you follow.

This document was prepared by Jason Sarich with key contributions from
Roscoe Bartlett, Michael A. Heroux, Barry Smith, and James M.
Willenbring.

\end{document}
